\levkolonttl{О. А. МАТВЕЕВА}

\prvkolonttl{О НУЛЯХ ПОЛИНОМОВ ДИРИХЛЕ, АППРОКСИМИРУЮЩИХ \ldots}

\thispagestyle{empty}

\input{shapka.tex}

\setcounter{equation}{0}

\setcounter{theorem}{0}

\setcounter{lemm}{0}

\setcounter{corollary}{0}

\setcounter{footnote}{0}

\setcounter{section}{0}
УДК 511.3
\begin{center}
{\Large \bf О НУЛЯХ ПОЛИНОМОВ ДИРИХЛЕ, АППРОКСИМИРУЮЩИХ В КРИТИЧЕСКОЙ

\medskip
ПОЛОСЕ L-ФУНКЦИИ ДИРИХЛЕ}

\medskip
{\large О.~А.~Матвеева (г. Саратов)}
\end{center}

\newpage

\begin{abstract}
Получены плотностные теоремы о нулях полиномов Дирихле,
аппроксимирующих L-функции Дирихле в критической области.

\medskip
Ключевые слова: полиномы Дирихле, L-функции Дирихле, нули полиномов
Дирихле.
\end{abstract}
\begin{center}
{\Large \bf ZEROS OF DIRICHLET POLYNOMIALS

\medskip
APPROXIMATING DIRICHLET L-FUNCTIONS

\medskip
IN THE CRITICAL STRIP}

\medskip
{\large O.~A.~Matveeva}
\end{center}

\begin{engabstract}
Density theorems about zeros of dirichlet polynomials approximating
Dirichlet L-fuctions in the critical strip are obtained.

\medskip
Key words: Dirichlet polynomials, Dirichlet L-fuctions, zeros of
Dirichlet polynomials.
\end{engabstract}

\section{Введение}
В работе \cite{KorotkovMatveeva} была приведена вычислительная схема
построения полиномов Дирихле  $Q_n(s),\; s = \sigma + it$, которые
в прямоугольнике $0 < \sigma < 1,\; 0 < t < T$ аппроксимируют целые
функции, заданные рядами Дирихле с периодическими коэффициентами,
с показательной скоростью. В частности, эта схема позволяет
эффективно вычислять нули L-функций Дирихле, лежащие в
критической полосе. В данной работе показано, что, с одной стороны,
известные факты о нулях L-функций Дирихле дают возможность получить
результаты о нулях аппроксимирующих полиномов Дирихле; с другой
стороны, поведение в критической полосе аппроксимирующих полиномов
Дирихле определяет поведение L-функций Дирихле.

\newpage
\section{Конструкция полиномов Дирихле, аппроксимирующих
в критической полосе L-функции Дирихле}
Рассмотрим L-функцию Дирихле
\begin{equation}
L(s, \chi) = \sum_{1}^{\infty}\frac{\chi(n)}{n^s}, \quad
s = \sigma + it,
\label{Meq10}
\end{equation}
и соответствующий степенной ряд
\begin{equation}
\label{powerseries}
g(z) = \sum_{1}^{\infty}\chi(n)t^n.
\end{equation}

.....

Для оценки величины \eqref{Meq10} сверху необходимо применить
численную схему, которая связана с вычислением полиномов $Q_n(s)$.

В заключении отметим, что аналогичные факты будут иметь место
и в случае рядов
Дирихле с периодическими коэффициентами.
\begin{thebibliography}{99}
\bibitem{KorotkovMatveeva} Коротков, А. Е. Об одном численном
алгоритме определения нулей целых функций, определяемых рядами
Дирихле с периодическими коэффициентами / А. Е. Коротков,
О. А. Матвеева // Научные ведомости Белгородского государственного
университета. Сер. Математика. Физика.~--- Белгород: Изд-во
НИУ "Белгу", 2011. Вып.24, \No 17(112). С. 47-53.

.....

\bibitem{Prahar}
Прахар К. Распределение простых чисел / К. Прахар.~--- М.:
Мир, 1967.~--- 513 с.
\end{thebibliography}

\noindent Саратовский государственный университет им.
Н. Г. Чернышевского.

\noindent Получено 10.03.2013
